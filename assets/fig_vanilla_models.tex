

\begin{figure}[!thbp] % 浮动位置建议
\centering
\caption{This is the performance comparison for the two vanilla models.} % 您可以修改为您的总标题
\label{fig:A_spans_rows}

% 使用 tabularray 环境来创建网格布局
% colspec: 定义三列,X[c,m] 表示列宽自动调整、内容水平居中(c)、垂直居中(m)
% rowsep: 行间距
\begin{tblr}{
  colspec = {X[c,m] X[c,m] X[c,m]},
  rowsep = {0.5em} % 行之间的间距,可以调整
}
% --- 第一行 ---
\SetCell[r=2]{c} % 设置此单元格 (图片A) 跨越2行 (r=2),内容居中 (c)
  \begin{subfigure}{\linewidth} % 子图A,宽度为当前单元格宽度
    \centering
    % 占位符A,高度设置得较大,宽度自适应单元格
    \fbox{\rule[-.5cm]{0cm}{6cm} \rule[-.5cm]{0.9\linewidth}{0cm}}
    \caption{Performance metrics Comparison (the heat map)}
    \label{fig:sub_A_span}
  \end{subfigure}
& % 第二列,第一行 (图片B)
  \begin{subfigure}{\linewidth}
    \centering
    \fbox{\rule[-.5cm]{0cm}{2.8cm} \rule[-.5cm]{0.9\linewidth}{0cm}}
    \caption{Confusion matrix for logistic regression}
    \label{fig:sub_B}
  \end{subfigure}
& % 第三列,第一行 (图片C)
  \begin{subfigure}{\linewidth}
    \centering
    \fbox{\rule[-.5cm]{0cm}{2.8cm} \rule[-.5cm]{0.9\linewidth}{0cm}}
    \caption{Decision boundary for logistic regression}
    \label{fig:sub_C}
  \end{subfigure}
\\ % 结束第一行 (B和C所在的行)

% --- 第二行 ---
% 第一列由图片A跨行占据,所以这里为空
& % 第二列,第二行 (图片D)
  \begin{subfigure}{\linewidth}
    \centering
    \fbox{\rule[-.5cm]{0cm}{2.8cm} \rule[-.5cm]{0.9\linewidth}{0cm}}
    \caption{Confusion matrix for decision tree}
    \label{fig:sub_D}
  \end{subfigure}
& % 第三列,第二行 (图片E)
  \begin{subfigure}{\linewidth}
    \centering
    \fbox{\rule[-.5cm]{0cm}{2.8cm} \rule[-.5cm]{0.9\linewidth}{0cm}}
    \caption{Decision boundary for decision tree}
    \label{fig:sub_E}
  \end{subfigure}
\\ % 结束第二行
\end{tblr}
\end{figure}

