
%\begin{document}


\begin{figure}[!htbp] % 使用 [htbp] 作为浮动位置建议
  \centering % 整体居中所有子图

  % 第一行
  \begin{subfigure}{0.3\textwidth} % 每个子图大约占页面宽度的30%
    \centering
    \fbox{\rule[-.5cm]{0cm}{3cm} \rule[-.5cm]{3cm}{0cm}} % 调整了占位符大小以便排列
    \caption{A}
    \label{fig:sub1}
  \end{subfigure}\hfill % \hfill 用于在子图之间平均分配水平空间
  \begin{subfigure}{0.3\textwidth}
    \centering
    \fbox{\rule[-.5cm]{0cm}{3cm} \rule[-.5cm]{3cm}{0cm}}
    \caption{B}
    \label{fig:sub2}
  \end{subfigure}\hfill
  \begin{subfigure}{0.3\textwidth}
    \centering
    \fbox{\rule[-.5cm]{0cm}{3cm} \rule[-.5cm]{3cm}{0cm}}
    \caption{C}
    \label{fig:sub3}
  \end{subfigure}

  \vspace{1em} % 在两行子图之间添加一些垂直间距

  % 第二行
  \begin{subfigure}{0.3\textwidth}
    \centering
    \fbox{\rule[-.5cm]{0cm}{3cm} \rule[-.5cm]{3cm}{0cm}}
    \caption{D}
    \label{fig:sub4}
  \end{subfigure}\hfill
  \begin{subfigure}{0.3\textwidth}
    \centering
    \fbox{\rule[-.5cm]{0cm}{3cm} \rule[-.5cm]{3cm}{0cm}}
    \caption{E}
    \label{fig:sub5}
  \end{subfigure}\hfill
  \begin{subfigure}{0.3\textwidth}
    \centering
    \fbox{\rule[-.5cm]{0cm}{3cm} \rule[-.5cm]{3cm}{0cm}}
    \caption{F}
    \label{fig:sub6}
  \end{subfigure}

  \caption{This figure shows the visualization results using t-SNE with different dimensionality and different perplexity. The corresponding parameters are showed on the figure.}
  \label{fig:main_figure_grid}
\end{figure}


%\end{document}