\begin{table}[htbp] % 建议使用 [htbp] 而不是 [!h] 以获得更好的浮动效果
\centering
\caption{The relationship between performance of t-SNE (measured with KL divergence) and the target dimension and perplexity. The lower the KL divergence value is, the better the performace is.} % 精简后的标题
\label{tab:tsne_performance_comparison} % 添加一个标签方便引用
\begin{tabular}{@{}llccc@{}} % @{} 移除了表格边缘的额外空白,调整了列格式
\toprule
& & \multicolumn{3}{c}{Perplexity} \\
\cmidrule(lr){3-5} % (lr) 表示左右稍微缩短线条,只覆盖 Perplexity 下的列
& & 10 & 30 & 50 \\
\midrule
\multirow{2}{*}{Dimensionality} & 2 & ValueA & ValueB & ValueC \\ % 用占位符 ValueA 等代替实际数据
& 3 & ValueD & ValueE & ValueF \\ % 用占位符 ValueD 等代替实际数据
\bottomrule
\end{tabular}
\end{table}