\documentclass{article}


% if you need to pass options to natbib, use, e.g.:
%     \PassOptionsToPackage{numbers, compress}{natbib}
% before loading neurips_2025


% ready for submission
\usepackage[preprint]{neurips_2025}


% to compile a preprint version, e.g., for submission to arXiv, add add the
% [preprint] option:
%     \usepackage[preprint]{neurips_2025}


% to compile a camera-ready version, add the [final] option, e.g.:
%     \usepackage[final]{neurips_2025}


% to avoid loading the natbib package, add option nonatbib:
%    \usepackage[nonatbib]{neurips_2025}


\usepackage[utf8]{inputenc} % allow utf-8 input
\usepackage[T1]{fontenc}    % use 8-bit T1 fonts
\usepackage{hyperref}       % hyperlinks
\usepackage{url}            % simple URL typesetting
\usepackage{booktabs}       % professional-quality tables
\usepackage{amsfonts}       % blackboard math symbols
\usepackage{amsmath}
\usepackage{nicefrac}       % compact symbols for 1/2, etc.
\usepackage{microtype}      % microtypography
\usepackage{xcolor}         % colors
\usepackage{multicol}
\usepackage{multirow}
\usepackage{graphicx} % 虽然现在是占位符,但将来替换图片时会用到
\usepackage{subcaption} % 用于创建子图环境
\usepackage{makecell}
\usepackage{tabularray}

\title{Project Report \large \\ DSAA2011 Machine Learning}


% The \author macro works with any number of authors. There are two commands
% used to separate the names and addresses of multiple authors: \And and \AND.
%
% Using \And between authors leaves it to LaTeX to determine where to break the
% lines. Using \AND forces a line break at that point. So, if LaTeX puts 3 of 4
% authors names on the first line, and the last on the second line, try using
% \AND instead of \And before the third author name.


\author{%
  % examples of more authors
   Hua XU \\
   HKUST(GZ) \\
   \texttt{hxu401@connect.hkust-gz.edu.cn} \\
   \And
   Jianhao RUAN \\
   HKUST(GZ) \\
   \texttt{jruan189@connect.hkust-gz.edu.cn} \\
   \And
   Leyi WU \\
   HKUST(GZ) \\
   \texttt{lwu398@connect.hkust-gz.edu.cn} \\
}


\begin{document}


\maketitle

\section{Introduction}

\underline{don't forget to finish this part}


\section{Methods and results}

\underline{add an introductory sentence here}

\subsection{Data preprocessing}

\begin{itemize}
    \item \underline{observed patterns of dataset or insights}
\end{itemize}

The Adult Income datasets is a binary classification dataset with 14 usable variables, six of which are integer types while the rest eight variables are categorical. In this project, we are using different strategy to process the data.

\begin{itemize}
    \item As for the categorical variables, we firstly use the most frequent value to impute the missing values in the data, then we use one-hot encoding to convert the values into one-hot vectors. This step is implemented in Python with \texttt{OneHotEncoder} in scikit-learn.
    \item As for the numeric variables, we use median values of the variables to replace the missing values, and after dealing with the missing values, we then normalize the data to $0$ mean and $1$ variance with \texttt{StandardScaler} in scikit-learn.
\end{itemize}

After imputing missing values and converting all features to numeric ones, we then get a wide dataframe with 105 features, making the computation costs high and taking much time in training and testing. For efficiency and visualization purposes, we then use t-SNE to reduce the dimensionality of the original data and get both the 2D and 3D embedding for future usages.

\begin{table}[htbp] % 建议使用 [htbp] 而不是 [!h] 以获得更好的浮动效果
\centering
\caption{The relationship between performance of t-SNE (measured with KL divergence) and the target dimension and perplexity. The lower the KL divergence value is, the better the performace is.} % 精简后的标题
\label{tab:tsne_performance_comparison} % 添加一个标签方便引用
\begin{tabular}{@{}llccc@{}} % @{} 移除了表格边缘的额外空白,调整了列格式
\toprule
& & \multicolumn{3}{c}{Perplexity} \\
\cmidrule(lr){3-5} % (lr) 表示左右稍微缩短线条,只覆盖 Perplexity 下的列
& & 10 & 30 & 50 \\
\midrule
\multirow{2}{*}{Dimensionality} & 2 & ValueA & ValueB & ValueC \\ % 用占位符 ValueA 等代替实际数据
& 3 & ValueD & ValueE & ValueF \\ % 用占位符 ValueD 等代替实际数据
\bottomrule
\end{tabular}
\end{table}

\underline{justify why we should use t-sne in 3D: comparing the time and performance}


\subsection{Data visualization}

\underline{explicitly show the mis assigned labels and try to make the 3D graph more readable}


%\begin{document}


\begin{figure}[!htbp] % 使用 [htbp] 作为浮动位置建议
  \centering % 整体居中所有子图

  % 第一行
  \begin{subfigure}{0.3\textwidth} % 每个子图大约占页面宽度的30%
    \centering
    \fbox{\rule[-.5cm]{0cm}{3cm} \rule[-.5cm]{3cm}{0cm}} % 调整了占位符大小以便排列
    \caption{A}
    \label{fig:sub1}
  \end{subfigure}\hfill % \hfill 用于在子图之间平均分配水平空间
  \begin{subfigure}{0.3\textwidth}
    \centering
    \fbox{\rule[-.5cm]{0cm}{3cm} \rule[-.5cm]{3cm}{0cm}}
    \caption{B}
    \label{fig:sub2}
  \end{subfigure}\hfill
  \begin{subfigure}{0.3\textwidth}
    \centering
    \fbox{\rule[-.5cm]{0cm}{3cm} \rule[-.5cm]{3cm}{0cm}}
    \caption{C}
    \label{fig:sub3}
  \end{subfigure}

  \vspace{1em} % 在两行子图之间添加一些垂直间距

  % 第二行
  \begin{subfigure}{0.3\textwidth}
    \centering
    \fbox{\rule[-.5cm]{0cm}{3cm} \rule[-.5cm]{3cm}{0cm}}
    \caption{D}
    \label{fig:sub4}
  \end{subfigure}\hfill
  \begin{subfigure}{0.3\textwidth}
    \centering
    \fbox{\rule[-.5cm]{0cm}{3cm} \rule[-.5cm]{3cm}{0cm}}
    \caption{E}
    \label{fig:sub5}
  \end{subfigure}\hfill
  \begin{subfigure}{0.3\textwidth}
    \centering
    \fbox{\rule[-.5cm]{0cm}{3cm} \rule[-.5cm]{3cm}{0cm}}
    \caption{F}
    \label{fig:sub6}
  \end{subfigure}

  \caption{This figure shows the visualization results using t-SNE with different dimensionality and different perplexity. The corresponding parameters are showed on the figure.}
  \label{fig:main_figure_grid}
\end{figure}


%\end{document}

With 2D and 3D t-SNE embedding, we would like to visualize the dataset and explore more information within the dataset. For the 2D visualization, the first thing we notice is that samples from two different classes are not very distinguishable as they fall in the same clusters with different labels. As for samples with income less than $50$K, they are spread over almost all the figure while the samples with income greater than $50$K tend to cluster on only one side of the
figure, which indicates that the class $\le 50K$ is a more general, diverse class with samples from different distribution while for the class $>50K$. 

If we take a closer look at the labels inside, we may see that there are different sub clusters.

Moreover, this also indicates that most of the features in the original dataset may not be useful in terms of separating the samples apart and these features make the samples close to each other in the original feature space, and we may need to find the important ones for better performance.

Moreover, this also indicates that if we just directly use the t-SNE embedding to perform future tasks, we may fail since we cannot separate the samples based on the embedding good enough.

\subsection{Clustering analysis}

\begin{table}[htbp] % Changed from [!h] to [htbp] for better float placement
    \centering % Center the table on the page
    \caption{Comparison of Clustering Algorithm Performance Metrics} % Added a descriptive caption
    \label{tab:clustering_performance} % Added a label for referencing
    \begin{tabular}{@{}l cccc@{}} % @{} removes extra padding at the ends of the table
        \toprule
        \thead{Algorithm} & \thead{Silhouette \\ Score} & \thead{Calinski-Harabasz \\ Score} & \thead{Davies-Bouldin \\ Score} & \thead{Adjusted Rand \\ Score} \\
        \midrule
        K-means                  & 0.123  & 12345.67 & 1.234  & 0.123 \\ % Placeholder data
        Agglomerative Clustering & 0.456  & 67890.12 & 0.987  & 0.456 \\ % Placeholder data
        % Add more rows as needed
        \bottomrule
    \end{tabular}
\end{table}

\begin{figure}[!htbp] % 使用 [!htbp] 作为浮动位置建议 (与原代码一致)
  \centering % 整体居中所有子图

  % 一行,两个子图
  \begin{subfigure}{0.48\textwidth} % 每个子图大约占页面宽度的48%
    \centering
    \fbox{\rule[-.5cm]{0cm}{3cm} \rule[-.5cm]{3cm}{0cm}} % 占位符,尺寸可以根据需要调整
    \caption{A} % 第一个子图的标题
    \label{fig:subA_1x2} % 第一个子图的标签 (建议修改标签以反映新布局)
  \end{subfigure}\hfill % \hfill 用于在子图之间平均分配水平空间
  \begin{subfigure}{0.48\textwidth} % 第二个子图也大约占页面宽度的48%
    \centering
    \fbox{\rule[-.5cm]{0cm}{3cm} \rule[-.5cm]{3cm}{0cm}} % 占位符
    \caption{B} % 第二个子图的标题
    \label{fig:subB_1x2} % 第二个子图的标签
  \end{subfigure}
  % 由于只有一行,不再需要 \vspace{1em} 和第二行的代码

  \caption{This figure shows the clustering results for K-means and agglomerative hierachical clustering.} % 主图的标题 (保持不变)
  \label{fig:main_figure_1x2_grid} % 主图的标签 (建议修改标签以反映新布局)
\end{figure}

\subsubsection*{Methods}

Based on the t-SNE visualization, we have already observed the clustering characteristic within the data and we can perform clustering analysis to gain more insights. Here we are using K-means and agglomerative hierarchical clustering.

\paragraph{K-means} K-means is a classic clustering algorithm, which iteratively calculates the distance between sample points and the centroids of the cluster. The algorithm can be divided into \textbf{assignment step} and \textbf{update step}, as the assignment step assigns each data point $x_i$ from the dataset $X$ to the closest cluster centroid $\mu_i$ using the formula $r_i = \underset{k\in \{1\ldots K\} }{\arg\min}{ \|x_{i}-\mu_k\| }$. The centroids will then be updated during the
update step, as each of the centroid of the cluster will be updated with the formula $\mu_k = \frac{\sum_{i:r_i=k}^{} x_i}{\sum_{i:r_i=k}^{} 1}$. In this method, the choose of the initial centroids is very important and therefore here we uses the \textbf{k-means++} initialization, which choose the first center randomly from the data points and for each remaining point, compute its squared distances to the nearest existing center. After that we choose another centroid with
probability proportional to the squared distance. Then we repeat the process until $K$ centers are chosen. This can reduce sensitivity to poor initialization, theoretically guarantees $O(\log K)$-competitive solutions. We implement this method in Python with \texttt{KMeans} in scikit-learn.

\paragraph{Agglomerative clustering} Agglomerative clustering is a bottom-up approach which starts with each data point as a single cluster and merges the closest pairs of clusters iteratively. It will continue the process until one cluster remains. In this method, the way to measure the distance between the clusters will affect the performance of clustering methods, and we may have for different ways: 1) single linkage, which uses minimum pairwise distance; 2) complete linkage,
which uses maximum pairwise distance; 3) average linkage, which uses average of all pairwise distances; 4) wardd's method, which aims at minimizing within-cluster sum of squares measured by the costs of merging cluster $A$ and $B$: $\Delta(A, B)=\frac{2n_An_b}{n_A+n_B}\|\mu_A-\mu_B\|^2$ where $n_A, n_B$ means the number of elements in the corresponding clusters. We implement this method in Python with \texttt{AgglomerativeClustering} in scikit-learn.

For these two methods, k-means is the most classic clustering methods which can act as a \underline{baseline} and since it is fast, it can quickly bring us some insights on the clustering analysis. And since we have already see sub-clusters in the t-SNE analysis, we may try to use hierarchical clustering for more precisely capturing the locality of small sub-clusters.

\subsubsection*{Results}

The clustering results is visualized in \underline{Fig} and we have used some methods to evaluate the performance of the clustering.

\underline{need to analyze the results and tell which one is better in terms of performance}

\underline{need to gain more insights from the clustering}


\subsection{Prediction: training and testing}

\underline{need to describe the chosen classification target, model classes and why they were selected}

\underline{need to describe the training process}


\subsubsection*{Methods}

Based on the t-SNE visualization we may see that it is very hard to find a continuous line or surface to separate the samples from the two different classes apart. 
Therefore, as for prediction, we would like to use decision tree based models and logistic regressions, as the decision tree based models may create discrete decision boundaries while logistic regression may bring us a simple decision boundary, making the classification precise.

\paragraph{Logistic regression} Logistic regressions 123456.

\paragraph{Decision trees} Decision trees 123456.

\subsubsection*{Results}



\begin{figure}[!thbp] % 浮动位置建议
\centering
\caption{This is the performance comparison for the two vanilla models.} % 您可以修改为您的总标题
\label{fig:A_spans_rows}

% 使用 tabularray 环境来创建网格布局
% colspec: 定义三列,X[c,m] 表示列宽自动调整、内容水平居中(c)、垂直居中(m)
% rowsep: 行间距
\begin{tblr}{
  colspec = {X[c,m] X[c,m] X[c,m]},
  rowsep = {0.5em} % 行之间的间距,可以调整
}
% --- 第一行 ---
\SetCell[r=2]{c} % 设置此单元格 (图片A) 跨越2行 (r=2),内容居中 (c)
  \begin{subfigure}{\linewidth} % 子图A,宽度为当前单元格宽度
    \centering
    % 占位符A,高度设置得较大,宽度自适应单元格
    \fbox{\rule[-.5cm]{0cm}{6cm} \rule[-.5cm]{0.9\linewidth}{0cm}}
    \caption{Performance metrics Comparison (the heat map)}
    \label{fig:sub_A_span}
  \end{subfigure}
& % 第二列,第一行 (图片B)
  \begin{subfigure}{\linewidth}
    \centering
    \fbox{\rule[-.5cm]{0cm}{2.8cm} \rule[-.5cm]{0.9\linewidth}{0cm}}
    \caption{Confusion matrix for logistic regression}
    \label{fig:sub_B}
  \end{subfigure}
& % 第三列,第一行 (图片C)
  \begin{subfigure}{\linewidth}
    \centering
    \fbox{\rule[-.5cm]{0cm}{2.8cm} \rule[-.5cm]{0.9\linewidth}{0cm}}
    \caption{Decision boundary for logistic regression}
    \label{fig:sub_C}
  \end{subfigure}
\\ % 结束第一行 (B和C所在的行)

% --- 第二行 ---
% 第一列由图片A跨行占据,所以这里为空
& % 第二列,第二行 (图片D)
  \begin{subfigure}{\linewidth}
    \centering
    \fbox{\rule[-.5cm]{0cm}{2.8cm} \rule[-.5cm]{0.9\linewidth}{0cm}}
    \caption{Confusion matrix for decision tree}
    \label{fig:sub_D}
  \end{subfigure}
& % 第三列,第二行 (图片E)
  \begin{subfigure}{\linewidth}
    \centering
    \fbox{\rule[-.5cm]{0cm}{2.8cm} \rule[-.5cm]{0.9\linewidth}{0cm}}
    \caption{Decision boundary for decision tree}
    \label{fig:sub_E}
  \end{subfigure}
\\ % 结束第二行
\end{tblr}
\end{figure}




After fitting the model with the train data, we then get the trained model and the corresponding decision boundary is also visualized in the \underline{figure}. From the figure we can see that the decision tree model is more expressive, as it can bring us multiple decision areas and decision boundaries while the logistic regression model cannot handle the complex data distribution and its decision boundary and the decision boundary just go cross some of the samples. 


This intuition implied by the visualizatoin is also supported by the metrics. From the table \underline{below} we can see that as for the overall performance. 

\underline{interpretion of results}


However, it is quite clear that the performance is still not amazingly great, as the accuracy is just around \underline{1234} which is far below the legend of tree based models. Since we think the power of tree models is from its ability to create flexible decision areas, we think that adding the depth of the trees or increasing the number of trees in the process of fitting may be a good idea. We then increase the depth of a single decision tree to \underline{1245} and did more experiments,
while we used the ensemble models to do more experiments. Their decision boundary are visualized in \underline{figure xxx} and the performance metrics are in \underline{table xxx}. Based on the results above, we may see that the ability of classification does increase as the tree goes deeper, 
\underline{(but as the tree goes too deep, the performance on the test set will decrease as the model start to overfit)}. 
And for the decision boundaries, we can see that the two models can bring us more complex but robust decision boundaries.

\input{assets/fig_ensemble_models}

Also, we may see the importance of features for further support on our conclusion and the results can be referred to in \underline{fig xxx}. For the best performance model XGBoost, we can see that the importance of features is significantly imbalanced as the \underline{xxxxx} matters most while the model pays little attention on the other features. For the vanilla decision tree model and the random forest model, they pay attention to more features but result in worse performance. This
interesting discovery align with the mechanism of XGBoost and the rest, as the gradient boosting model will try to learn from its failures and may be more able to learn the features which can make it away from error, while the random forest or the vanilla decision models will pay equal attention on all of the features and will be more likely affected by noises. This fact also aligns to our t-SNE visualization in \underline{figure} as it clearly shows the existence of noisy features and
only a few of features really matters.

Meanwhile, the performance on one class is significantly worse than another class, which may due to the not balanced distribution in the train data.

\underline{check whether we have already explicitly compare the model performance}

\subsubsection*{Model selection}

\input{assets/fig_model_selection}

After fitting the models, we can select the model to use by analyzing their performance. The classification results are illustrated in \underline{table xxxxxxx} and based on that, we can calculate the ROC and AUC in \underline{fig xxx}. It shows that model \underline{xxxx} has the greatest AUC and it can balance well on the data


\section{Conclusion}

In this project, we try to explore the intrinsic characteristics of the Adult Income dataset and try to perform classification tasks with different models on it. Through our analysis, we find that the dataset contains a lot of redundant features which do not help the task too much but still will be involved in the classification process, making the costs big and the performance bad. Only a few features matter. For better time efficiency, we firstly perform dimensionality reduction
with t-SNE and found that the classes are hard to separate and there are some sub-clusters in the dataset. 

\underline{then we visualized the dataset with clustering methods}

Then we start to perform classification task on it. For the complex data distribution, logistic regression perform badly on the dataset as it cannot create flexible decision boundary. Tree based models performs good on the dataset, as the performance will increase as the trees get deeper. Ensemble models is the king on this task, among those the XGBoost is the good good winner, as which can learn from its failures and find the most important feature. However, the model still suffer from the
imbalanced distribution of the two labels and in the future, more advanced techniques should be applied for better performance.

\underline{we may need to find one or two papers to support our conclusion that only the specific feature is important on the dataset}





\label{conclusion}

\section*{References}
{
\small


[1] Alexander, J.A.\ \& Mozer, M.C.\ (1995) Template-based algorithms for
connectionist rule extraction. In G.\ Tesauro, D.S.\ Touretzky and T.K.\ Leen
(eds.), {\it Advances in Neural Information Processing Systems 7},
pp.\ 609--616. Cambridge, MA: MIT Press.


[2] Bower, J.M.\ \& Beeman, D.\ (1995) {\it The Book of GENESIS: Exploring
  Realistic Neural Models with the GEneral NEural SImulation System.}  New York:
TELOS/Springer--Verlag.


[3] Hasselmo, M.E., Schnell, E.\ \& Barkai, E.\ (1995) Dynamics of learning and
recall at excitatory recurrent synapses and cholinergic modulation in rat
hippocampal region CA3. {\it Journal of Neuroscience} {\bf 15}(7):5249-5262.
}

\section*{Credit}
\begin{itemize}
    \item \textbf{Leyi WU}: 
    \item \textbf{Hua XU}: 
    \item \textbf{Jianhao RUAN}:
    \item \textbf{AI}: Formatting latex tables and figures.

\end{itemize}




\end{document}
